\documentclass[10pt]{beamer}

\usetheme[progressbar=frametitle]{metropolis}

\usepackage{xeCJK}
\usepackage{booktabs}
\usepackage[scale=2]{ccicons}
\usefonttheme[onlymath]{serif}

\usepackage{pgfplots}
\usepgfplotslibrary{dateplot}

\usepackage{xspace}
\newcommand{\themename}{\textbf{\textsc{metropolis}}\xspace}


% Informotion
%-------------------
\title{LinearLang}
\subtitle{A Memory Safety Language Without Garbage Collection.}
\date{\today}
\author{王璐璐, 朱思文, 白宗磊}
\institute{Peking University}
\titlegraphic{\hfill\includegraphics[height=2cm]{pku}} % "logo" comes from "pku.pdf"
\begin{document}
\maketitle


% START
%-------------------
\begin{frame}{Table of contents}
  \setbeamertemplate{section in toc}[sections numbered]
  \tableofcontents[hideallsubsections]
\end{frame}



% section 1 -------------------------------------------
\section{Motivation}

%page 1.1
\begin{frame}[fragile]{Memory Error}
    \metroset{block=fill}
	\begin{block}{Different types of memory errors}
		\begin{itemize}
			\item dangling pointer / wild pointer
			\item memory leak
			\item double free
			\item ...
		\end{itemize}
	\end{block}
\end{frame}


% page 1.2
\begin{frame}[fragile]{background}
  Two main approaches for memory management.

      \metroset{block=fill}

      \begin{exampleblock}{Manual memory management}
      	\begin{itemize}
      		\item \textbf{Pros}: high performance. 
      		\item \textbf{Cons}: not safe, memory leak
      	\end{itemize}
      \end{exampleblock}


      \begin{exampleblock}{Garbage collection}
      	\begin{itemize}
      		\item \textbf{Pros}: (sumi)automatic, safe. 
      		\item \textbf{Cons}: low performance, memory leak
      	\end{itemize}
      \end{exampleblock}
\end{frame}


% page 1.3
\begin{frame}[fragile]{example1}
\textbf{dangling pointer}
\begin{verbatim}
function writeTwoObjs(obj1, obj2){
    obj1.f()
    obj2.f()
    delete obj1
    delete obj2   //error!
}

let obj = new Object()
writeTwoObjs(obj,obj)
\end{verbatim}
\end{frame}


%page 1.4
\begin{frame}[fragile]{example2}
\textbf{memory leak}
\begin{verbatim}
function sendMsg(str, addr){
    let s = new socket(addr)
    write(s,str)
    //close(s)
}

sendMsg("hello",addr1)
sendMsg("world",addr2)
\end{verbatim}
\end{frame}



% section 2 -------------------------------------------
\section{Linear type system}


%page 2.1
\begin{frame}[fragile]{linear type system}
    \metroset{block=fill}
	
	\begin{block}{background}
	\begin{itemize}
		\item based on linear logic a substructural logic proposed by Jean-Yves Girard
	\end{itemize}
	\end{block}
	
	\begin{block}{linear type system}
	\begin{itemize}
		\item each linear variable must be used(transfer ownership/delete) exactly once
		\item no aliasing is possible for linear resource
	\end{itemize}
	\end{block}

	\begin{block}{related}
	\begin{itemize}
		\item \textbf{C++1x}: smart pointer: unique\_ptr, shared\_ptr
		\item \textbf{rust}:  ownership, move, borrow
	\end{itemize}
	\end{block}
\end{frame}




% page 2.2
\begin{frame}[fragile]{Our Work}
	Based on linear type system we designed a programming language that do not have garbage collection and satisfies:
	\metroset{block=fill}
	\begin{block}{A well typed program will never have:}
		\begin{itemize}
			\item dangling pointer
			\item memory leak
		\end{itemize}
	\end{block}
\end{frame}


% Section 3 -------------------------------------------
\section{Language and typing rules}


% page 3.1
\begin{frame}[fragile]{Memory Model}
	\metroset{block=fill}
	\begin{block}{Two kinds of value:}
		\begin{itemize}
			\item \textbf{basic value:} call by value, stored in stack, released automatically
			\item \textbf{resource value:} call by reference, stored in heap, released manually
		\end{itemize}
	\end{block}
	
	\begin{block}{stack and heap}
		\begin{itemize}
			\item \textbf{stack:} list of (variable,  basic value | address)
			\item \textbf{heap:} list of (address, resource) 
		\end{itemize}	
	\end{block}

\end{frame}


%page3.2
\begin{frame}[fragile]{Overview}
	\textbf{judgment}
		\[
			\Gamma \vdash term : Ty
		\]
		\[
			\Downarrow
		\]
		\[
			\Gamma_{in} \vdash term : Ty\ ;\Gamma_{out}
		\]
\textbf{in practice:}		
		\[
			type\_of\ (term,\Gamma_{in}) \rightarrow (T,\Gamma_{out})
		\]
\end{frame}

%page 3.3
\begin{frame}[fragile]{basic terms}
\begin{verbatim}
type term =
    | Num_term       of int
    | Add_term       of term * term
    | Bool_term      of bool     
    | Le_term        of term * term
    | Eq_term        of term * term
    | And_term       of term * term
    | Or_term        of term * term
    | Not_term       of term
    | Var_term       of string
    | If_term        of term * term * term
    | Lambda_term    of string * ty * term
    | App_term       of term * term 
    | Fix_term       of term
    | Begin_term     of term list
    ...
\end{verbatim}
\end{frame}


\begin{frame}[fragile]{types}
\begin{verbatim}
type ty =
    | NumTy
    | BoolTy
    
    | LinResTy
    | LinListTy
    
    | UnitTy
    | ArrowTy   of ty * ty
\end{verbatim}
\end{frame}


\begin{frame}[fragile]{basic typing rules}
\textbf{UnVar:}
\[
	\Gamma_1 , x : NumTy, \Gamma_2 \vdash x : NumTy; \Gamma_1 , x : NumTy, \Gamma_2
\]
\[
	\Gamma_1 , x : BoolTy, \Gamma_2 \vdash x : BoolTy; \Gamma_1 , x : BoolTy, \Gamma_2
\]
\textbf{LinVar}
\[
	\Gamma_1 , x : LinResTy, \Gamma_2 \vdash x : LinResTy; \Gamma_1 , \Gamma_2
\]
\[
	\Gamma_1 , x : LinListTy, \Gamma_2 \vdash x : LinListTy; \Gamma_1 , \Gamma_2
\]
\end{frame}


\begin{frame}[fragile]{more typing rules}
\textbf{If}
\[
	\frac{
		\begin{array}{c}
			\Gamma_1 \vdash tm1 : BoolTy\ ; \Gamma_2 \\
			\Gamma_2 \vdash tm2: T\ ; \Gamma_3 \\
			\Gamma_2 \vdash tm3: T\ ; \Gamma_3
		\end{array}
	}{
		\Gamma_1 \vdash if\ tm1\ then\ tm2\ else\ tm3\ : T\ ; \Gamma_3
	}
\]
\textbf{Abs:}
\[
	\frac{
		\begin{array}{c}
			\Gamma_1, x:T_1 \vdash tm : T_2 ; \Gamma_2 \\
			\Gamma_1 = \Gamma_2 \div x
		\end{array}
	}{
		\Gamma_1 \vdash fun(x:T_1)\{tm\} : T_1 \rightarrow T_2\ ; \Gamma_2 \div x
	}
\]
\end{frame}

%page 3.4
\begin{frame}[fragile]{two kinds of let binding}
\metroset{block=fill}
\begin{verbatim}
type term =
    ...
    | LetUn_term       of string * term * term
    | LetLin_term      of string * term * term 
   
    | NewLinRes_term   of string
    ...
\end{verbatim}

\begin{exampleblock}{example}
\begin{verbatim}
letUn x = 123
letLin y = newRes("aaa")
\end{verbatim}	
\end{exampleblock}

\begin{alertblock}{type error!}
\begin{verbatim}
letLin x = 123
letUn y = newRes("aaa")
\end{verbatim}	
\end{alertblock}

\end{frame}



\begin{frame}[fragile]{typing rules of let binding}
\textbf{letLin:}
\[
	\frac
	{
		\begin{array}{c}
			\Gamma_1 \vdash tm_1 : T_1 ; \Gamma_2 \\
			T_1 = LinResTy\ \vee T_1 = LinListTy \\
			\Gamma_2, x:T_1 \vdash tm_2 : T_2 ; \Gamma_3			
		\end{array}
	}
	{
		\Gamma_1 \vdash letUn\ x = tm_1\ in\ tm_2 : T_2 ; \Gamma_3 \div x
	}
\]

\textbf{letUn:}
\[
	\frac
	{
		\begin{array}{c}
			\Gamma_1 \vdash tm_1 : T_1 ; \Gamma_2 \\
			\neg (T_1 = LinResTy\ \vee T_1 = LinListTy) \\
			\Gamma_2, x:T_1 \vdash tm_2 : T_2 ; \Gamma_3			
		\end{array}
	}
	{
		\Gamma_1 \vdash letLin\ x = tm_1\ in\ tm_2 : T_2 ; \Gamma_3 \div x
	}
\]
\end{frame}



%page 3.5
\begin{frame}[fragile]{linear list}
\metroset{block=fill}
\begin{verbatim}
type term =
    ...
    | LinCons_term      of term * term
    | Null_term
    | Split_term        of term * string * string * term
    | FreeAtom_term     of term
    | FreeList_term     of term
    ...
\end{verbatim}

\begin{exampleblock}{example}
\begin{verbatim}
letLin x = newRes("aaa")
letLin y = newRes("bbb")
letLin l = linCons(x,linCons(y,null))
split l as carx cdrx
freeAtom(carx)
freeList(cdrx)
\end{verbatim}	
\end{exampleblock}
\end{frame}


\begin{frame}[fragile]{typing rules of linear list}
\textbf{linCons:}
\[
	\frac{
		\begin{array}{c}
			 \Gamma_1 \vdash tm1 : T_1 ; \Gamma_2 \\
			 \Gamma_2 \vdash tm2 : T_2 ; \Gamma_3 \\
			 T_1 = LinResTy \\bigwedge T_2 = LinListTy			
		\end{array}
	}{
		\Gamma_1 \vdash linCons(tm1,tm2) : LinListTy\ ; \Gamma_3
	}
\]
\textbf{split:}
\[
	\frac{
		\begin{array}{c}
			\Gamma_1 \vdash tm1 : T_1 ; \Gamma_2 \\
			T_1 = LinListTy \\
			\Gamma_2, tcar : LinResTy, tcdr : LinListTy \vdash tm2 : T_2 ; \Gamma_3			
		\end{array}
	}{
		\Gamma_1 \vdash split\ tm1\ as\ tcar\ tcdr\ in\ tm2: T_2 ; (\Gamma_3 \div tcar) \div tcdr
	}
\]
\textbf{freeAtom}
\[
	\frac{
		\begin{array}{c}
			\Gamma_1 \vdash tm : LinResTy\ ; \Gamma_2
		\end{array}
	}{
		\Gamma_1 \vdash freeAtom(tm) : UnitTy\ ; \Gamma_2
	}
\]
\end{frame}

%page 3.6
\begin{frame}[fragile]{helpful primitive operations}
\begin{verbatim}
type term = 
    ...
    | CopyAtom_term     of term * string * string * term
    | CopyList_term     of term * string * string * term
    | If_null_term      of string * term * term
    | AppendList_term   of term * term
    | Print_term        of term
\end{verbatim}

\end{frame}



\begin{frame}[fragile]{example}
\metroset{block=fill}
\begin{alertblock}{type error}
\begin{verbatim}
letLin x = newRes("aaa")
letLin y = f(x)
letLin l = linCons(x,linCons(y,null))   //type error!
\end{verbatim}	
\end{alertblock}

\begin{exampleblock}{well typed}
\begin{verbatim}
letLin x = newRes("aaa")
copyAtom x as x1 x2
letLin y = f(x1)
letLin l = linCons(x2,linCons(y,null))   //well typed!
\end{verbatim}	
\end{exampleblock}
\end{frame}



\begin{frame}[fragile]{typing rules}
\textbf{copyAtom}
\[
	\frac{
		\begin{array}{c}
			\Gamma_1 \vdash tm1 : LinResTy\ ; \Gamma_2 \\
			\Gamma_2, v1: LinResTy, v2: LinResTy \vdash tm2 : T\ ; \Gamma_3
		\end{array}
	}{
		\Gamma_1 \vdash copyAtom\ tm1\ as\ v1\ v2\ in\ tm2 : T\ ; (\Gamma_3 \div v1) \div v2
	}
\]
\textbf{IfNull:}
\[
	\frac{
		\begin{array}{c}
			\Gamma_1 \vdash var : LinListTy\ ; \Gamma_2 \\
			\Gamma_2, var : LinListTy \vdash tm2: T\ ; \Gamma_3 \\
			\Gamma_2, var : LinListTy \vdash tm3: T\ ; \Gamma_3
		\end{array}
	}{
		\Gamma_1 \vdash if\_null\ var\ then\ tm2\ else\ tm3\ : T\ ; \Gamma_3
	}
\]

\end{frame}



% Section 4 -------------------------------------------
\section{Demo}

\begin{frame}{reference}
	\begin{itemize}
		\item Pierce B C. Types and Programming Languages. 
		\item Pierce B C. Advanced Topics in Types and Programming Languages 
		\item F Pottier. Wandering through linear types, capabilities, and regions
		\item Harvard University. CS 152: Programming Languages
		\item Baker H G. A “linear logic” Quicksort[M]. ACM, 1994
		\item Baker, Henry G. "Lively linear Lisp:“look ma, no garbage!”." Acm Sigplan Notices 27.8(1992):89-98.
	\end{itemize}
\end{frame}


\begin{frame}{小组分工}
	\begin{itemize}
		\item 查找资料: 朱思文,白宗磊
		\item 语言定义: 朱思文,白宗磊,王璐璐
		\item typing rules: 白宗磊,王璐璐
		\item operational semantics: 朱思文,白宗磊
		\item 编程实现: 王璐璐
		\item 课堂报告: 王璐璐
		\item 书面报告: 朱思文, 白宗磊 
	\end{itemize}
\end{frame}

\begin{frame}[standout]
	Thank you!
\end{frame}

% page 4.1

%------------------------------

\end{document}
