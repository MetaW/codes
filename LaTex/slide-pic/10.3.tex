\documentclass{beamer}
\usepackage{xeCJK}
\usepackage{graphicx}

\usetheme{Darmstadt}
%\usetheme{default}

\title[标题]{Timed automata}
\author[作者]{金钊,王璐璐,白宗磊}
\institute[单位]{北京大学信息科学与技术学院}
\date[日期]{2017/6/1}

\begin{document}

%------------------------------------------------

\begin{frame}
    \titlepage
\end{frame}

%------------------------------------------------

\begin{frame}{Outline}
    \begin{itemize}
   		\item 10.1 Motivation
   		\item 10.2 Syntax of timed automata
   		\item \textbf{10.3 Semantics of timed automata}
   		\item 10.4 Networks of timed automata
   		\item 10.5 More on timed automata formalisms
    \end{itemize}
\end{frame}

%------------------------------------------------

\section{10.3 Semantics of timed automata}

\begin{frame}
	\begin{center}
		\item 10.3 Semantics of timed automata
	\end{center}
\end{frame}


\begin{frame}
	\frametitle{state}
	\begin{block}{state}
		\begin{itemize}
			\item a suitable notion of the state of computation of a timed automaton consists of a pair $(l,v)$
			\item $l$ is the control location the automaton is in
			\item $v$ is the valuation determined by the current clock values
		\end{itemize}		
	\end{block}
\end{frame}


\begin{frame}
	\frametitle{timed transition system}
	\begin{block}{timed transition system $T(A)$}
		Let $A = (L, l_0, E, $I$)$ be a timed automaton over a set of clocks $C$ and a set of actions $Act$. We define the timed transition system $T(A)$ generated by $A$ as $T(A) = (Proc, Lab, \{ \stackrel{\alpha}{\longrightarrow} | \alpha \in Lab \})$
	\end{block}
\end{frame}


\begin{frame}
	\frametitle{timed transition system}
	\begin{block}{timed transition system $T(A)$}
		其中:
		\begin{itemize}
			\item $Proc = \{ (l,v)| (l,v) \in L * (C \rightarrow R_{\geq 0}) \quad and \quad v\vDash $I$(l) \}$. states are of the form $(l, v)$, where $l$ is a location of the timed automaton and $v$ is a valuation that satisfies the invariant of $l$.
			\item $Lab = Act \cup R_{\geq 0}$ is the set of labels $(R_{\geq 0}: time - elapsing \quad step)$
			\item the transition relation is defined by:
				\begin{itemize}
					\item $(l,v) \stackrel{a}{\longrightarrow} (l^{'},v^{'})$ if there is an edge $ (l \stackrel{g,a,r}{\longrightarrow} l^{'}) \in E $ such that $v \vDash g , v^{'} = v[r] \quad and \quad v^{'} \vDash $I$(l^{'})$
					\item $(l,v) \stackrel{d}{\longrightarrow} (l,v + d)$ for all $d \in R_{\geq 0}$ such that $ v \vDash $I$(l) \quad and \quad v+d \vDash $I$(l)$
				\end{itemize}
		\end{itemize}
	\end{block}
			where: g is the guard, a is the action, r is the set of clocks to be reset, $I$ assigns invariants to locations.
\end{frame}


\begin{frame}
	\frametitle{timed transition system}
	\begin{block}{timed transition system $T(A)$}
		Let $v_0$ denote the valuation such that $v_0 (x) = 0$ for all $x ∈\in C$ . If $v_0$ satisfies the invariant of the initial location $l_0$, we shall call $(l_0 , v_0 )$ the initial state (or initial configuration ) of $T(A)$.
	\end{block}
\end{frame}


\begin{frame}
	\frametitle{Example}
	\begin{center}
		\includegraphics[width=0.6\textwidth]{./pp1.png}
	\end{center}
	A small part of the transition system $T(A)$ is shown below (there are in fact uncountably many different reachable states for every $x$ in the interval [0, 2])
	\begin{center}
		\includegraphics[width=1.0\textwidth]{./pp2.png}
	\end{center}
\end{frame}


\begin{frame}
	\frametitle{Notice:}
	There is a fundamental difference between situations where a clock constraint is used in the guard and where it is used in the invariant.
	\begin{center}
		\includegraphics[width=0.8\textwidth]{./pp3.png}
	\end{center}
\end{frame}


\begin{frame}
	\frametitle{Notice:}
	\begin{center}
		\includegraphics[width=0.6\textwidth]{./pp4.png}
	\end{center}
	In the timed automaton (b), $x <= 1$ is used in the invariant. This means that it is never possible to delay more than 1 time unit
\end{frame}


\begin{frame}
	\frametitle{Exercise:}
	\begin{center}
		\includegraphics[width=0.6\textwidth]{./pp5.png}
	\end{center}
	A Worker alternates between resting and working. The clock x is used for constraining the time spent by the Worker in these two modes, and the clock y is used to control the frequency with which the Worker is hitting nails while working. 
\end{frame}



\begin{frame}
	\frametitle{}
	\begin{center}
		谢谢大家!
	\end{center}
\end{frame}

%------------------------------------------------


\end{document}