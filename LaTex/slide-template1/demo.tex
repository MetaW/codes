\documentclass[10pt]{beamer}

\usetheme[progressbar=frametitle]{metropolis}

\usepackage{booktabs}
\usepackage[scale=2]{ccicons}

\usepackage{pgfplots}
\usepgfplotslibrary{dateplot}

\usepackage{xspace}
\newcommand{\themename}{\textbf{\textsc{metropolis}}\xspace}


% Informotion
%-------------------
\title{Amazing Report}
\subtitle{This is really exciting}
\date{\today}
\author{Wang Lulu}
\institute{Peking University}
\titlegraphic{\hfill\includegraphics[height=1.5cm]{logo}} % "logo" comes from "logo.pdf"
\begin{document}
\maketitle


% START
%-------------------
\begin{frame}{Table of contents}
  \setbeamertemplate{section in toc}[sections numbered]
  \tableofcontents[hideallsubsections]
\end{frame}



% section 1 -----
\section{Introduction}



% page 1.1
\begin{frame}[fragile]{Metropolis}

  The \themename theme is a Beamer theme with minimal visual noise
  inspired by the \href{https://github.com/hsrmbeamertheme/hsrmbeamertheme}{\textsc{hsrm} Beamer
  Theme} by Benjamin Weiss.

  Enable the theme by loading \ldots

  \begin{verbatim}    
	\documentclass{beamer}
	\usetheme{metropolis}
  \end{verbatim}

  Note, that you have to have Mozilla's \emph{Fira Sans} font and XeTeX
  installed to enjoy this wonderful typography.
\end{frame}



% section 2 -----
\section{Elements}



% page 2.1
\begin{frame}[fragile]{Typography}
	\begin{verbatim}
		The theme provides sensible defaults to \emph{emphasize} 
		text, \alert{accent} parts or show \textbf{bold} results.
	\end{verbatim}

	\begin{center}
		normal
		\emph{emphasize}
		\alert{alert}
		\textbf{bold}
	\end{center}
	
	\begin{quote}
    Quoted things should be write in this place.
  	\end{quote}

\end{frame}



% page 2.2
\begin{frame}{Font feature test}
  \begin{itemize}
    \item Regular
    \item \textit{Italic}
    \item \textsc{SmallCaps}
    \item \textbf{Bold}
    \item \textbf{\textit{Bold Italic}}
    \item \textbf{\textsc{Bold SmallCaps}}
    \item \texttt{Monospace}
    \item \texttt{\textit{Monospace Italic}}
    \item \texttt{\textbf{Monospace Bold}}
    \item \texttt{\textbf{\textit{Monospace Bold Italic}}}
  \end{itemize}
\end{frame}



% page 2.2
\begin{frame}{Lists}
  \begin{columns}[T,onlytextwidth]
  
	\column{0.33\textwidth}
		This is items
		\begin{itemize}
			\item Milk 
			\item Eggs 
			\item Potatos
		\end{itemize}

    \column{0.33\textwidth}
		This is enumerations
		\begin{enumerate}
        	\item First, 
        	\item Second 
        	\item Last.
		\end{enumerate}

    \column{0.33\textwidth}
		This is descriptions
      	\begin{description}
        	\item[PowerPoint] Meeh. 
        	\item[Beamer] Yeeeha.
      	\end{description}
      	
  \end{columns}
\end{frame}




% page 2.3
\begin{frame}{Animation}
  \begin{itemize}[<+- | alert@+>] % this is important
  
    \item This is important
    \item Now this
    \item And now this
    
  \end{itemize}
\end{frame}




% page 2.4
\begin{frame}{Tables}
  
	\begin{table}[tbp]
	\centering  
	\begin{tabular}{c|c|c|c|c}
		\hline
		the value of $k$ &$k=m$ &$k=0.1*m$ &$k=0.01*m$ &$k=0.005*m$\\ 
		\hline  
		accuracy of test set &99.47\% &\textbf{97.25\%} &85.56\% &82.11\%\\     
		training time & 1543.8s &\textbf{7.2s} &0.62s & 0.27s \\
		memory requirements & 765.81kb & \textbf{76.59kb} & 7.69kb & 3.84kb \\    
		\hline
	\end{tabular}
	\caption{This is a table}
	\end{table}

\end{frame}



% page 2.5
\begin{frame}{Blocks}
  Three different block environments are pre-defined and may be styled with an
  optional background color.

      \metroset{block=fill}

      \begin{block}{Default}
        Block content.
      \end{block}

      \begin{alertblock}{Alert}
        Block content.
      \end{alertblock}

      \begin{exampleblock}{Example}
        Block content.
      \end{exampleblock}

\end{frame}



% page 2.6
\begin{frame}{Math}
	\[
			D({x^{(1)}},{x^{(2)}},...,{x^{(m)}}) = \sum\limits_{i = 1}^m {{{\left\| {{x^{(i)}} - x_n^{(ci)}} \right\|}^2}} 
	\]
\end{frame}



% page 2.7
\begin{frame}[fragile]{Frame footer}
    this page has a footer.\footnote{this is a footer}
\end{frame}


% page 2.8
\begin{frame}{References}

	Write the information of reference paper in demo.bib 
	and only cited with cite command in this slide, the paper 
	can be find in the later "Reference" part, for example:

	\cite{knuth92,ConcreteMath,Simpson,Er01,greenwade93}
	
\end{frame}




% Section 3 -----
\section{Conclusion}

% page 3.1
\begin{frame}{license}

	Get the source of this theme and the demo presentation from

	\begin{center}
  		\url{github.com/matze/mtheme}
	\end{center}

  	The theme \emph{itself} is licensed under a
  	\href{http://creativecommons.org/licenses/by-sa/4.0/}{Creative Commons
  	Attribution-ShareAlike 4.0 International License}.

  	\begin{center}
  		\ccbysa
  	\end{center}
\end{frame}



% page 3.2 
\begin{frame}[standout]
  This page is excited! 
\end{frame}

%------------------------------


\appendix


% reference

\begin{frame}[allowframebreaks]{References}

  \bibliography{demo} % demo comes from the file "demo.bib"
  
  \bibliographystyle{abbrv}

\end{frame}




\end{document}
